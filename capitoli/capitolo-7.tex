% !TEX encoding = UTF-8
% !TEX TS-program = pdflatex
% !TEX root = ../tesi.tex

%**************************************************************
\chapter{Conclusioni}
\label{cap:conclusioni}
%**************************************************************

%**************************************************************
\section{Raggiungimento degli obiettivi}
La pianificazione in termine di ore totali è stata perfettamente rispettata: lo stage si è svolto in 310 ore, rispettando le scadenze settimanali pianificate nella sezione \ref{sec:vincoli-temporali}. Ciò, purtroppo, significa che non c'è stato spazio per lo svolgimento delle attività \textbf{opzionali} descritte sezione \ref{section:altri-interventi-minori}.\\Riassumendo, lo stato di soddisfacimento degli obiettivi è stato dunque il seguente:
\begin{center}
	\def\arraystretch{1.5}
	\begin{longtable}{p{1.5cm} p{8.5cm} p{1.5cm} } 
		\hline
		\multicolumn{2}{c}{\textbf{Obbligatori}} & Soddisfatto\\
		\hline
		Ob1 & Interazione con il database SQL Server attraverso le librerie del \gls{framework} Codeigniter & Si\\
		\hline
		Ob2 & Realizzazione integrazione flat-file di un nuovo fornitore con il Data Exchange del \bookingEngine & Si\\
		\hline
		Ob3 & Aggiunta prodotti e tariffe del nuovo fornitore ai risultati della ricerca lato Front-End del \bookingEngine & Si\\
		\hline
		Ob4 & Esecuzione test e redazione documentazione sul lavoro svolto & Si\\
		\hline
		\multicolumn{2}{c}{\textbf{Desiderabili}} & Soddisfatto\\
		\hline
		D1 & Realizzazione del registro carichi/scarichi tariffe “vuoto per pieno” come	funzionalità lato Back-End del \bookingEngine & Si\\
		\hline
		D2 & Interrogazione web-service in tempo reale per sincronizzare prezzi e disponibilità del nuovo fornitore con il Data Exchange & Si\\
		\hline
		D3 & Realizzazione conferma prenotazione al fornitore come funzionalità lato Front-End del \bookingEngine & Si\\
		\hline
		\multicolumn{2}{c}{\textbf{Opzionali}} & Soddisfatto\\
		\hline
		Op1 & Analisi e realizzazione di nuove funzionalità & No\\
		\hline
	\end{longtable}
\end{center}
Avendo raggiunto tutti gli obiettivi obbligatori e desiderabili prefissati, considero l'esito dello stage più che soddisfacente, anche perché buona parte del codice da me sviluppato è stato portato direttamente in produzione.
%**************************************************************
\section{Competenze acquisite}
\subsection{Competenze tecnologiche}
A livello tecnologico, non ho acquisito grandi nuove conoscenze, data anche la mia esperienza pregressa in questo campo.\\ Ho avuto modo di interagire con il \gls{framework} \textit{Codeigniter}, che devo dire ho imparato ad apprezzare molto, soprattutto per la sua semplicità. Posso sicuramente affermare che riutilizzerò questo \gls{framework} nei prossimi progetti in PHP che andrò a realizzare. Ho avuto anche l'occasione di interfacciarmi con \textit{Microsoft SQL Server} che però, ai fin dei conti, non differisce poi più di tanto rispetto agli altri \gls{DBMS} da me già utilizzati in passato.
\subsection{Competenze metodologiche}
A livello metodologico, invece, posso affermare di aver ricevuto un grande valore aggiunto da questa esperienza. Ho potuto "toccare con mano" cosa significhi lavorare con metodo, ed i vantaggi che ciò porta a livello di produttività. Infatti, avere del metodo di lavoro, permette di realizzare prodotti aventi più alta qualità, che rispondono in modo più affine alle esigenze del committente. Ho anche potuto riscontrare come il modello evolutivo descritto in sezione \ref{sec:modello-evolutivo} permetta di rispondere molto bene alle esigenze mutevoli del mercato (anzi, sarebbe meglio dire del committente tipicamente indeciso). Posso dunque affermare di aver acquisito una \textit{way-of-working} che sicuramente andrò ad applicare ai miei progetti "amatoriali" che sto attualmente portando avanti.
%**************************************************************
\section{Valutazione sul rapporto azienda-università}
Ho iniziato a programmare da completo autodidatta. La scuola superiore che ho frequentato ha contribuito a darmi un'infarinatura accademica riguardo il mondo dell'informatica, ma sono sempre stato un cosiddetto "smanettone". Per imparare qualcosa, ho sempre prediletto l'aspetto pratico a quello teorico, perché permette di ricevere subito un frutto remunerativo del proprio lavoro. All'inizio del percorso della laurea triennale, devo dire di essere rimasto un po' deluso dall'eccessivo (a mio parere dell'epoca) approccio teorico nello studio di questa materia. Con l'avanzare del tempo, però, ho imparato che (almeno il più delle volte) è utile capire quello che si sta facendo, e per capirlo è necessario valutare anche l'approccio teorico al problema che si sta cercando di risolvere.\\
A posteriori, ora che, con la discussione di questo documento, ho concluso il mio percorso di laurea triennale, penso che l'approccio teorico offerto da questo corso di laurea possa giovare a tanti "smanettoni" come me, anche perché il mondo dell'informatica si evolve ad una velocità molto elevata, che rende molto difficile starne al passo. Ritengo tuttavia sarebbe una buona idea avviare collaborazioni, nel limite del possibile, con "entità esterne all'aula" (aziende, istituzioni) per lo svolgimento dei progetti tipici di alcuni corsi (ad esempio Tecnologie Web). Così facendo, probabilmente, si avrebbero degli studenti più abili nella risoluzione di problemi "reali", preparati ad affrontare il mondo del lavoro e sarebbe data loro la possibilità di esplorare tecnologie magari innovative ma sempre affini al corso (ad esempio Node.js), un po' come accade già con Ingegneria del Software. Probabilmente, infine, sarebbe anche più motivante per gli studenti stessi, in quanto realizzerebbero qualcosa di non fine a se stesso.