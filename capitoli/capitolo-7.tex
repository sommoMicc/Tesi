% !TEX encoding = UTF-8
% !TEX TS-program = pdflatex
% !TEX root = ../tesi.tex

%**************************************************************
\chapter{Conclusioni}
\label{cap:conclusioni}
%**************************************************************

%**************************************************************
\section{Raggiungimento degli obiettivi}
La pianificazione in termine di ore totali è stata perfettamente rispettata: lo stage si è svolto in 310 ore, rispettando le scadenze settimanali pianificate nella sezione \ref{sec:vincoli-temporali}. Ciò, purtroppo, significa che non c'è stato spazio per lo svolgimento delle attività \textbf{opzionali} descritte sezione \ref{section:altri-interventi-minori}.\\Riassumendo, lo stato di soddisfacimento degli obiettivi è stato dunque il seguente:
\begin{center}
	\def\arraystretch{1.5}
	\begin{longtable}{p{1.5cm} p{8.5cm} p{1.5cm} } 
		\hline
		\multicolumn{2}{c}{\textbf{Obbligatori}} & Soddisfatto\\
		\hline
		Ob1 & Interazione con il database SQL Server attraverso le librerie del framework Codeigniter & Si\\
		\hline
		Ob2 & Realizzazione integrazione flat-file di un nuovo fornitore con il Data Exchange del \bookingEngine & Si\\
		\hline
		Ob3 & Aggiunta prodotti e tariffe del nuovo fornitore ai risultati della ricerca lato Front-End del \bookingEngine & Si\\
		\hline
		Ob4 & Esecuzione test e redazione documentazione sul lavoro svolto & Si\\
		\hline
		\multicolumn{2}{c}{\textbf{Desiderabili}} & Soddisfatto\\
		\hline
		D1 & Realizzazione del registro carichi/scarichi tariffe “vuoto per pieno” come	funzionalità lato Back-End del \bookingEngine & Si\\
		\hline
		D2 & Interrogazione web-service in tempo reale per sincronizzare prezzi e disponibilità del nuovo fornitore con il Data Exchange & Si\\
		\hline
		D3 & Realizzazione conferma prenotazione al fornitore come funzionalità lato Front-End del \bookingEngine & Si\\
		\hline
		\multicolumn{2}{c}{\textbf{Opzionali}} & Soddisfatto\\
		\hline
		Op1 & Analisi e realizzazione di nuove funzionalità & No\\
		\hline
	\end{longtable}
\end{center}
Avendo raggiunto tutti gli obiettivi obbligatori e desiderabili prefissati, considero l'esito dello stage più che soddisfacente, anche perché buona parte del codice da me sviluppato è stato portato direttamente in produzione.
%**************************************************************
\section{Competenze acquisite}
\subsection{Competenze tecnologiche}
A livello tecnologico, non ho acquisito grandi nuove conoscenze, data anche la mia esperienza pregressa in questo campo.\\ Ho avuto modo di interagire con il framework \textit{Codeigniter}, che devo dire ho imparato ad apprezzare molto, soprattutto per la sua semplicità. Posso sicuramente affermare che riutilizzerò questo framework nei prossimi progetti in PHP che andrò a realizzare. Ho avuto anche l'occasione di interfacciarmi con \textit{Microsoft SQL Server} che però, ai fin dei conti, non differisce poi più di tanto rispetto agli altri \gls{DBMS} da me già utilizzati in passato.
\subsection{Competenze metodologiche}
A livello metodologico, invece, posso affermare di aver ricevuto un grande valore aggiunto da questa esperienza. Ho potuto "toccare con mano" cosa significhi lavorare con metodo, ed i vantaggi che ciò porta a livello di produttività. Infatti, avere del metodo di lavoro, permette di realizzare prodotti aventi più alta qualità, che rispondono in modo più affine alle esigenze del committente. Ho anche potuto riscontrare come il modello evolutivo descritto in sezione \ref{sec:modello-evolutivo} permetta di rispondere molto bene alle esigenze mutevoli del mercato (anzi, sarebbe meglio dire del committente tipicamente indeciso). Posso dunque affermare di aver acquisito una \textit{way-of-working} che sicuramente andrò ad applicare ai miei progetti "amatoriali" che sto attualmente portando avanti.
%**************************************************************
\section{Valutazione personale}
