% !TEX encoding = UTF-8
% !TEX TS-program = pdflatex
% !TEX root = ../tesi.tex

%**************************************************************
\chapter{Descrizione dello stage}
\label{cap:descrizione-stage}
%**************************************************************

\section{Vantaggi per l'azienda}
WebPD trae diversi vantaggi dall'attività di stage curricolare che è stata disposta ad ospitare.\\
\\
Primo su tutti, l'inserimento in azienda, seppur solo per un paio di mesi, di un nuovo membro del personale, mai entrato in contatto con l'azienda. Ciò, in primis, ha permesso di distribuire il carico di lavoro tra più persone, permettendo di accelerare lo sviluppo sui progetti in cantiere. Inoltre, l'introduzione nel team di una persona completamente esterna all'azienda, ha portato un ulteriore punto di vista all'interno del team di sviluppo. Tale punto di vista si è dimostrato utile nel tentativo di risoluzione di alcuni problemi software "cronici" (come la lentezza di esecuzione delle query, problema che verrà descritto nel dettaglio nel prossimo capitolo), permettendo un ragionamento fuori dagli schemi mentali dell'ideatore di tale software.\\
\\
In secondo luogo, ha permesso all'azienda di esplorare nuovi canoni stilistici per alcuni suoi prodotti a costo zero, come nel caso del restyling della homepage del sito CrociereRegalo (descritta anch'essa nel prossimo capitolo), senza quindi il rischio di sacrificare inutilmente il lavoro (e la retribuzione) di un membro del personale.

\section{Presentazione del progetto}
L'obiettivo di questo stage è stato permettere a WebPD di completare la riscrittura (da capo) del sito CrociereRegalo.it. Tale sito, infatti, prima dell'ingresso di Primarete tra le quote di WebPD, era stato realizzato e mantenuto da WebCola, una web agency con la quale Primarete aveva stretto una partnership commerciale, che si è appunto interrotta nel 2015.\\
Secondo gli accordi presi, il sorgente del sito era di proprietà di WebCola, pertanto non è stato possibile per WebPD procedere ad una semplice modifica/aggiornamento di qualcosa già esistente. Il vecchio sito, inoltre, non era responsive e, dai dati di Google Analytics é emerso che la maggior parte delle visite avveniva da dispositivi mobili. \\
La problematica maggiore, comunque, erano (e lo sono tuttora) gli accordi presi con le varie compagnie di crociera (MSC, Costa, Royal Caribbean, Celebrity e Azamara): tali accordi, infatti, erano stati presi da WebCola in nome e per conto suo, quindi WebPD si è vista obbligata a ristabilirli.
\begin{figure}[!h] 
	\centering 
	\includegraphics[width=.9\columnwidth]{stage/webcola_crociereregalo} 
	\caption{Screenshot della versione di CrociereRegalo sviluppata da WebCola}
\end{figure}\\
Dal 2015 fino ad luglio 2018, WebPD è riuscita a creare un Booking Engine che interagisse con \Gls{api} e \glspl{webservice} forniti da Costa e MSC, permettendo di acquistare una crociera, pagando tramite un payment gateway fornito dal consorzio TVB (Triveneto Bassilichi). La soluzione realizzata, però, aveva grossi problemi di prestazioni (ad esempio la homepage del sito aveva un tempo di risposta medio di 7 secondi, a cui poi doveva essere sommato il tempo di download della risposta, rendering grafico ed esecuzione del codice Javascript presente) e mancava di alcune funzionalità, come la possibilità di vendere tariffe \textit{vuoto per pieno}. Esse non sono altro posti (cabine) acquistati da un'agenzia viaggi (nel caso in esame, Primarete) ad un prezzo scontato e rivenduti poi ai privati.
\\
Premesso ciò, lo stage ha avuto come obiettivo il completamento e l'ampliamento delle funzionalità offerte dal Booking Engine alla base di CrociereRegalo. Più nello specifico, lo stage si proponeva di svolgere le seguenti attività: 
\begin{enumerate}
	\item Studio del funzionamento del Booking Engine, comprendente la presa di confidenza con il framework PHP \textit{Codeigniter} e con il DBMS \textit{Microsoft SQL Server};
	\item Ottimizzazione, tramite tool di SQL Server, dei database alla base del Booking Engine, attraverso la creazione guidata di indici per ottimizzare i tempi di esecuzione delle query più complesse;
	\item Ottimizzazione, tramite l'utilizzo delle funzionalità di caching delle query presente in Codeigniter, del tempo di caricamento delle varie pagine del sito (con particolare attenzione alla homepage);
	\item Integrazione delle tariffe \textit{vuoto per pieno} nel flusso di prenotazione;
	\item Aggiunta della possibilità di prenotare crociere Royal Caribbean, Celebrity e Azamara, tramite l'integrazione dei \glspl{webservice} forniti da Fibos, partner informatico di Royal Caribbean (e delle sue controllate Celebrity e Azamara);
	\item Restyling della homepage e di alcune pagine secondarie del sito, secondo i suggerimenti forniti dalla ditta che si occupa di ottimizzazione SEO.	
\end{enumerate}

\section{Aspettative aziendali}
In fase di definizione contenutistica dello stage, i punti sopra descritti sono stati distribuiti in obiettivi aventi tre livelli di priorità, tenendo conto anche del numero di ore ridotto (circa 300) a disposizione dello stagista, identificati dalle seguenti sigle:
\begin{itemize}
	\item \textbf{Ob} per i requisiti obbligatori, vincolanti in quanto obiettivo primario richiesto dal committente;
	\item \textbf{D} per i requisiti desiderabili, non vincolanti o strettamente necessari, ma dal riconoscibile valore aggiunto;
	\item \textbf{Op} per i requisiti opzionali, rappresentanti valore aggiunto non strettamente competitivo.
\end{itemize}

\begin{tabularx}{\textwidth}{|X|X|}
	\hline
	\multicolumn{2}{|c|}{\textbf{Obbligatori}}\\
	\hline
	Ob1 & Interazione con il database SQL Server attraverso le librerie del framework Codeigniter\\
	\hline
	Ob2 & Realizzazione integrazione flat-file di un nuovo fornitore con il Data Exchange del Booking Engine\\
	\hline
	Ob3 & Aggiunta prodotti e tariffe del nuovo fornitore ai risultati della ricerca lato Front-End del Booking Engine\\
	\hline
	Ob4 & Esecuzione test e redazione documentazione sul lavoro svolto\\
	\hline
	\hline
	\multicolumn{2}{|c|}{\textbf{Desiderabili}}\\
	\hline
	D1 & Realizzazione del registro carichi/scarichi tariffe “vuoto per pieno” come	funzionalità lato Back-End del Booking Engine\\
	\hline
	D2 & Interrogazione web-service in tempo reale per sincronizzare prezzi e disponibilità del nuovo fornitore con il Data Exchange\\
	\hline
	D3 & Realizzazione conferma prenotazione al fornitore come funzionalità lato Front-End del Booking Engine\\
	\hline
	\multicolumn{2}{|c|}{\textbf{Opzionali}}\\
	\hline
	Op1 & Analisi e realizzazione di nuove funzionalità\\
	\hline
\end{tabularx}
