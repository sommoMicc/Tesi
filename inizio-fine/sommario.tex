% !TEX encoding = UTF-8
% !TEX TS-program = pdflatex
% !TEX root = ../tesi.tex

%**************************************************************
% Sommario
%**************************************************************
\cleardoublepage
\phantomsection
\pdfbookmark{Sommario}{Sommario}
\begingroup
\let\clearpage\relax
\let\cleardoublepage\relax
\let\cleardoublepage\relax

\chapter*{Sommario}

Il presente documento relaziona il prodotto del lavoro, della durata di circa trecentodieci ore, svolto dal laureando Michele Tagliabue presso l'azienda WebPD s.r.l. durante il periodo di stage. Prima dell'inizio dell'attività sono stati fissati diversi obiettivi da raggiungere.\\
Per cominciare, si era pianificato di imparare ad interagire, tramite le librerie del framework Codeigniter, con il database relazionale SQL Server. 
Successivamente, si era prefissato di integrare nel Booking Engine un nuovo fornitore (Royal Caribbean). Nello specifico, si era prefissato di realizzare l'integrazione del flat-file (catalogo) fornito da Royal Caribbean, l'aggiunta delle crociere di Royal Caribbean ai risultati di ricerca, la comunicazione con i Web Service di Royal Caribbean per la sincronizzazione della disponibilità di cabine (e prezzi) e per la prenotazione di queste ultime.
Infine, era stato previsto lo sviluppo di un sistema di registro "carichi/scarichi" (magazzino) di tariffe "vuoto per pieno" e di eventuali funzionalità aggiuntive.\\ \\
Questa tesi si compone di quattro capitoli. Nel primo viene delineato il profilo dell'azienda e le metodologie di lavoro della stessa. Il secondo capitolo presenta (anche a livello tecnologico) il progetto al centro delle attività svolte durante lo stage, che verranno approfondite (suddivise in base agli obiettivi) nel terzo capitolo. Infine, il quarto capitolo presenta una valutazione retrospettiva del tirocinio, sia a livello oggettivo, considerando, ad esempio, il grado di soddisfacimento degli obiettivi, che soggettivo, esponendo, quindi, una mia valutazione personale su quanto svolto.

\section*{Convenzioni tipografiche}
Durante la stesura del testo sono state adottate le seguenti convenzioni tipografiche:
\begin{itemize}
	\item gli acronimi, le abbreviazioni e i termini ambigui o di uso non comune menzionati
	sono definiti nel glossario, situato alla fine del presente documento e ogni
	occorrenza è evidenziata in blu, come l'esempio seguente: \gls{DBMS};
	\item i termini in lingua straniera o facenti parti del gergo tecnico sono evidenziati con
	il carattere corsivo.
\end{itemize}

%\vfill
%
%\selectlanguage{english}
%\pdfbookmark{Abstract}{Abstract}
%\chapter*{Abstract}
%
%\selectlanguage{italian}

\endgroup			

\vfill

