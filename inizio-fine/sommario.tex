% !TEX encoding = UTF-8
% !TEX TS-program = pdflatex
% !TEX root = ../tesi.tex

%**************************************************************
% Sommario
%**************************************************************
\cleardoublepage
\phantomsection
\pdfbookmark{Sommario}{Sommario}
\begingroup
\let\clearpage\relax
\let\cleardoublepage\relax
\let\cleardoublepage\relax

\chapter*{Sommario}

Tale documento relaziona il prodotto del lavoro, della durata di circa trecentodieci ore, svolto dal laureando Michele Tagliabue presso l'azienda WebPD s.r.l. durante il periodo di stage. Prima dell'inizio dell'attività sono stati prefissati diversi obiettivi, che hanno rispecchiato le attività svolte in azienda dal laureando.\\ \\
In primo luogo si era pianificato di imparare ad interagire, tramite le librerie del framework Codeigniter, con il database relazionale SQL Server. \\
Successivamente, si era prefissato di integrare nel Booking Engine un nuovo fornitore (Royal Caribbean). Nello specifico, si era prefissato di realizzare l'integrazione del flat-file (catalogo) fornito da Royal Caribbean, l'aggiunta delle crociere di Royal Caribbean ai risultati di ricerca, la comunicazione con i Web Service di Royal Caribbean per la sincronizzazione della disponibilità di cabine (e prezzi) e per la prenotazione di queste ultime.\\
Infine, era stato previsto lo sviluppo di un sistema di registro "carichi/scarichi" (magazzino) di tariffe "vuoto per pieno" come funzionalità Backend del Booking Engine.
%\vfill
%
%\selectlanguage{english}
%\pdfbookmark{Abstract}{Abstract}
%\chapter*{Abstract}
%
%\selectlanguage{italian}

\endgroup			

\vfill

