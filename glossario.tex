
%**************************************************************
% Acronimi
%**************************************************************
\renewcommand{\acronymname}{Acronimi e abbreviazioni}

\newacronym[description={\glslink{apig}{Application Program Interface}}]
    {api}{API}{Application Program Interface}

\newacronym[description={\glslink{umlg}{Unified Modeling Language}}]
    {uml}{UML}{Unified Modeling Language}

%**************************************************************
% Glossario
%**************************************************************
%\renewcommand{\glossaryname}{Glossario}

\newglossaryentry{incremento}
{
    name=\glslink{incremento}{incrementi},
    text=incremento,
    sort=incremento,
    description={Procedere per aggiunta ad una base},
    plural=incrementi
}

\newglossaryentry{iterazione}
{
    name=\glslink{iterazione}{iterazioni},
    text=iterazione,
    sort=iterazione,
    description={Procedere per rivisitazioni (può includere un incremento o addirittura un decremento).\\L'iterazione è un processo di durata non terminabile (anche potenzialmente infinita).},
    plural=iterazioni
}

\newglossaryentry{ide}
{
	name=\glslink{ide}{ide},
	text=IDE,
	sort=IDE,
	description={Un ambiente di sviluppo integrato (in lingua inglese integrated development environment ovvero IDE, anche integrated design environment o integrated debugging environment, rispettivamente ambiente integrato di progettazione e ambiente integrato di debugging), in informatica, è un software che, in fase di programmazione, aiuta i programmatori nello sviluppo del codice sorgente di un programma.\\ \\
	Spesso l'IDE aiuta lo sviluppatore segnalando errori di sintassi del codice direttamente in fase di scrittura, oltre a tutta una serie di strumenti e funzionalità di supporto alla fase di sviluppo e debugging.},
	plural=iterazioni
}