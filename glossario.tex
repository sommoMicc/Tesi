
%**************************************************************
% Acronimi
%**************************************************************
\renewcommand{\acronymname}{Acronimi e abbreviazioni}


\newacronym[description={\glslink{umlg}{Unified Modeling Language}}]
    {uml}{UML}{Unified Modeling Language}

%**************************************************************
% Glossario
%**************************************************************
%\renewcommand{\glossaryname}{Glossario}

\newglossaryentry{incremento}
{
    name=\glslink{incremento}{Incremento},
    text=incremento,
    sort=incremento,
    description={Procedere per aggiunta ad una base},
    plural=incrementi
}

\newglossaryentry{iterazione}
{
    name=\glslink{iterazione}{Iterazione},
    text=iterazione,
    sort=iterazione,
    description={Procedere per rivisitazioni (può includere un incremento o addirittura un decremento).\\L'iterazione è un processo di durata non terminabile (anche potenzialmente infinita)},
    plural=iterazioni
}

\newglossaryentry{ide}
{
	name=\glslink{ide}{IDE},
	text=IDE,
	sort=IDE,
	description={Un ambiente di sviluppo integrato (in lingua inglese integrated development environment ovvero IDE, anche integrated design environment o integrated debugging environment, rispettivamente ambiente integrato di progettazione e ambiente integrato di debugging), in informatica, è un software che, in fase di programmazione, aiuta i programmatori nello sviluppo del codice sorgente di un programma.\\ \\
	Spesso l'IDE aiuta lo sviluppatore segnalando errori di sintassi del codice direttamente in fase di scrittura, oltre a tutta una serie di strumenti e funzionalità di supporto alla fase di sviluppo e debugging\cite{site:wikipediaIDE}},
	plural=IDE
}


\newglossaryentry{api}
{
	name=\glslink{api}{API},
	text=API,
	sort=API,
	description={acronimo di Application Programming Interface. Serie di convenzioni adottate dagli sviluppatori di software per definire il modo con il quale va richiamata una determinata funzione di un'applicazione. L'impiego di API comuni ha lo scopo di rendere più omogenea l'interfaccia e di facilitare l'interazione di programmi che diversamente risulterebbero molto differenti e distanti fra loro\cite{site:api}},
	plural=API
}


\newglossaryentry{webservice}
{
	name=\glslink{webservice}{WebService},
	text=WebService,
	sort=WebService,
	description={Un WebService é un sistema software progettato per supportare interazioni macchina-macchina su una rete. Dispone di un'interfaccia descritta da un linguaggio processabile da una macchina (nello specifico WSDL). Altri sistemi interagiscono con il Web service in una maniera definita in base alla sua descrizione usando messaggi SOAP, tipicamente convogliati usando HTTP con serializzazione XML assieme ad altri standard Web\cite{site:webservice}},
	plural=WebServices
}


\newglossaryentry{seo}
{
	name=\glslink{seo}{SEO},
	text=SEO,
	sort=SEO,
	description={Acronimo di \textit{Search Engine Optimizazion}, definisce tutte le attività per migliorare il posizionamento di un determinato sito web nei motori di ricerca\cite{site:seo}},
	plural=SEO
}


\newglossaryentry{mvc}
{
	name=\glslink{mvc}{MVC},
	text=MVC,
	sort=MVC,
	description={Acronimo di \textit{Model View Controller}, è un design pattern architetturale in grado di separare la logica di presentazione dalla logica di business. Si compone di tre tipologie di componenti (classi): Modelli, che rappresentano i dati processati dall'applicazione, Viste che rappresentano l'interfaccia grafica dell'applicazione e Controller, che accetta in input il modello e lo converte in comandi per la vista\cite{book:mvc}},
	plural=MVC
}


\newglossaryentry{tempodirisposta}
{
	name=\glslink{tempodirisposta}{Tempo di risposta},
	text=tempo di risposta,
	sort=tempo di risposta,
	description={Tempo impiegato dal server per elaborare l'output, che verrà poi scaricato dal client. Il tempo di risposta, quindi, non include il tempo di download dell'output},
	plural=tempi di risposta
}

\newglossaryentry{tariffa}
{
	name=\glslink{tariffa}{Tariffa},
	text=tariffa,
	sort=tariffa,
	description={Gruppo di prezzi di una cabina, accomunati da uno o più fattori. La stessa cabina può avere due tariffe diverse (a prezzi diversi), perchè magari la prima include dei servizi che la seconda non ha (come bibite illimitate)},
	plural=tariffe
}

\newglossaryentry{RPC}
{
	name=\glslink{RPC}{RPC},
	text=RPC,
	sort=RPC,
	description={Remote Procedure Call, chiamata di una procedura remota\cite{site:rpc}},
	plural=RPC
}

\newglossaryentry{SOAP}
{
	name=\glslink{SOAP}{SOAP},
	text=SOAP,
	sort=SOAP,
	description={SOAP è un protocollo per lo scambio di messaggi tra componenti software, che permette di chiamare procedure remote (RPC Call, Remote Procedure Call). Richieste e risposte SOAP sono codificate con XML\cite{site:soap}},
	plural=SOAP
}

\newglossaryentry{DBMS}
{
	name=\glslink{DBMS}{DBMS},
	text=DBMS,
	sort=DBMS,
	description={Data Base Management System, sistema software progettato per la gestione (creazione, manipolazione, interrogazione) di basi di dati (database)\cite{site:dbms}},
	plural=DBMS
}

\newglossaryentry{RDBMS}
{
	name=\glslink{RDBMS}{RDBMS},
	text=RDBMS,
	sort=RDBMS,
	description={Relational Data Base Management System, DBMS basato sul modello relazionale\cite{site:rdbms}},
	plural=RDBMS
}


\newglossaryentry{framework}
{
	name=\glslink{framework}{framework},
	text=framework,
	sort=framework,
	description={Architettura software che include degli strumenti (classi, metodi) con lo scopo di semplificare lo sviluppo, facilitando così il lavoro del programmatore\cite{site:framework}},
	plural=framework
}


\newglossaryentry{wisp}
{
	name=\glslink{wisp}{WISP},
	text=WISP,
	sort=WISP,
	description={Acronimo di Windows (Server) - IIS - SQL Server - PHP. Viene utilizzato da WebPD per indicare lo stack tecnologico utilizzato dal \bookingEngine.},
	plural=WISP
}


\newglossaryentry{jquery}
{
	name=\glslink{jquery}{jQuery},
	text=jQuery,
	sort=jQuery,
	description={Una tra le più diffuse librerie Javascript, che agevola la manipolazione del \gls{dom}.},
	plural=jQuery
}

\newglossaryentry{dom}
{
	name=\glslink{dom}{DOM},
	text=DOM,
	sort=DOM,
	description={Acronimo di Document Object Model, rappresentazione in forma di albero di oggetti del contenuto della pagina HTML (Document) a cui si riferisce\cite{site:dom}}
}

\newglossaryentry{json}
{
	name=\glslink{json}{JSON},
	text=JSON,
	sort=JSON,
	description={Acronimo di JavaScript Object Notation, formato utilizzato per la rappresentazione di oggetti sotto forma di stringa\cite{site:json}},
	plural=JSON
}


\newglossaryentry{ajax}
{
	name=\glslink{ajax}{AJAX},
	text=AJAX,
	sort=AJAX,
	description={Acronimo di Asyncronous Javascript And XML, tecnica che prevede lo scambio di dati in background tra server e browser, utilizzando richieste HTTP asincrone\cite{site:ajax}},
	plural=AJAX
}
