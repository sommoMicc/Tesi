%**************************************************************
% file contenente le impostazioni della tesi
%**************************************************************

%**************************************************************
% Frontespizio
%**************************************************************

% Autore
\newcommand{\myName}{Michele Tagliabue}                                    
\newcommand{\myTitle}{Analisi, miglioramento e ampliamento delle funzionalità di un Booking Engine per la prenotazione di crociere}

% Tipo di tesi                   
\newcommand{\myDegree}{Tesi di laurea triennale}

% Università             
\newcommand{\myUni}{Università degli Studi di Padova}

% Facoltà       
\newcommand{\myFaculty}{Corso di Laurea in Informatica}

% Dipartimento
\newcommand{\myDepartment}{Dipartimento di Matematica "Tullio Levi-Civita"}

% Titolo del relatore
\newcommand{\profTitle}{Prof.}

% Relatore
\newcommand{\myProf}{Luigi De Giovanni}

% Luogo
\newcommand{\myLocation}{Padova}

% Anno accademico
\newcommand{\myAA}{2017-2018}

% Data discussione
\newcommand{\myTime}{Settembre 2018}

\newcommand{\bookingEngine}{\textit{Booking Engine}}

%**************************************************************
% Impostazioni di impaginazione
% see: http://wwwcdf.pd.infn.it/AppuntiLinux/a2547.htm
%**************************************************************

\setlength{\parindent}{14pt}   % larghezza rientro della prima riga
\setlength{\parskip}{0pt}   % distanza tra i paragrafi


%**************************************************************
% Impostazioni di biblatex
%**************************************************************
\bibliography{bibliografia} % database di biblatex 

\defbibheading{bibliography} {
    \cleardoublepage
    \phantomsection 
    \addcontentsline{toc}{chapter}{\bibname}
    \chapter*{\bibname\markboth{\bibname}{\bibname}}
}

\setlength\bibitemsep{1.5\itemsep} % spazio tra entry

\DeclareBibliographyCategory{opere}
\DeclareBibliographyCategory{web}

\addtocategory{opere}{womak:lean-thinking}
\addtocategory{web}{site:agile-manifesto}

\defbibheading{opere}{\section*{Riferimenti bibliografici}}
\defbibheading{web}{\section*{Siti Web consultati}}


%**************************************************************
% Impostazioni di caption
%**************************************************************
\captionsetup{
    tableposition=top,
    figureposition=bottom,
    font=small,
    format=hang,
    labelfont=bf
}

%**************************************************************
% Impostazioni di glossaries
%**************************************************************

%**************************************************************
% Acronimi
%**************************************************************
\renewcommand{\acronymname}{Acronimi e abbreviazioni}


\newacronym[description={\glslink{umlg}{Unified Modeling Language}}]
    {uml}{UML}{Unified Modeling Language}

%**************************************************************
% Glossario
%**************************************************************
%\renewcommand{\glossaryname}{Glossario}

\newglossaryentry{incremento}
{
    name=\glslink{incremento}{Incremento},
    text=incremento,
    sort=incremento,
    description={Procedere per aggiunta ad una base},
    plural=incrementi
}

\newglossaryentry{iterazione}
{
    name=\glslink{iterazione}{Iterazione},
    text=iterazione,
    sort=iterazione,
    description={Procedere per rivisitazioni (può includere un incremento o addirittura un decremento).\\L'iterazione è un processo di durata non terminabile (anche potenzialmente infinita).},
    plural=iterazioni
}

\newglossaryentry{ide}
{
	name=\glslink{ide}{IDE},
	text=IDE,
	sort=IDE,
	description={Un ambiente di sviluppo integrato (in lingua inglese integrated development environment ovvero IDE, anche integrated design environment o integrated debugging environment, rispettivamente ambiente integrato di progettazione e ambiente integrato di debugging), in informatica, è un software che, in fase di programmazione, aiuta i programmatori nello sviluppo del codice sorgente di un programma.\\ \\
	Spesso l'IDE aiuta lo sviluppatore segnalando errori di sintassi del codice direttamente in fase di scrittura, oltre a tutta una serie di strumenti e funzionalità di supporto alla fase di sviluppo e debugging.},
	plural=iterazioni
}


\newglossaryentry{api}
{
	name=\glslink{api}{API},
	text=API,
	sort=API,
	description={acronimo di Application Programming Interface. Serie di convenzioni adottate dagli sviluppatori di software per definire il modo con il quale va richiamata una determinata funzione di un'applicazione. L'impiego di API comuni ha lo scopo di rendere più omogenea l'interfaccia e di facilitare l'interazione di programmi che diversamente risulterebbero molto differenti e distanti fra loro.},
	plural=API
}


\newglossaryentry{webservice}
{
	name=\glslink{webservice}{WebService},
	text=WebService,
	sort=WebService,
	description={Un WebService é un sistema software progettato per supportare interazioni macchina-macchina su una rete. It has an interface described in a machine-processable format (specifically WSDL). Other systems interact with the Web service in a manner prescribed by its description using SOAP messages, typically conveyed using HTTP with an XML serialization in conjunction with other Web-related standards},
	plural=WebServices
}


\newglossaryentry{seo}
{
	name=\glslink{seo}{SEO},
	text=SEO,
	sort=SEO,
	description={Acronimo di \textit{Search Engine Optimizazion}, definisce tutte le attività per migliorare il posizionamento di un determinato sito web nei motori di ricerca},
	plural=SEO
}


\newglossaryentry{mvc}
{
	name=\glslink{mvc}{MVC},
	text=MVC,
	sort=MVC,
	description={Acronimo di \textit{Model View Controller}, è un design pattern architetturale in grado di separare la logica di presentazione dalla logica di business. Si compone di tre tipologie di componenti (classi): Modelli, che rappresentano i dati processati dall'applicazione, Viste che rappresentano l'interfaccia grafica dell'applicazione e Controller, che accetta in input il modello e lo converte in comandi per la vista.},
	plural=MVC
}


\newglossaryentry{tempodirisposta}
{
	name=\glslink{tempodirisposta}{Tempo di risposta},
	text=tempo di risposta,
	sort=tempo di risposta,
	description={Tempo impiegato dal server per elaborare l'output, che verrà poi scaricato dal client. Il tempo di risposta, quindi, non include il tempo di download dell'output.},
	plural=tempi di risposta
}

\newglossaryentry{tariffa}
{
	name=\glslink{tariffa}{tariffa},
	text=tariffa,
	sort=tariffa,
	description={Gruppo di prezzi di una cabina, accomunati da uno o più fattori. La stessa cabina può avere due tariffe diverse (a prezzi diversi), perchè magari la prima include dei servizi che la seconda non ha (come bibite illimitate).},
	plural=tariffe
}

\newglossaryentry{RPC}
{
	name=\glslink{RPC}{RPC},
	text=RPC,
	sort=RPC,
	description={Remote Procedure Call, chiamata di una procedura remota.},
	plural=RPC
}

\newglossaryentry{SOAP}
{
	name=\glslink{SOAP}{SOAP},
	text=SOAP,
	sort=SOAP,
	description={SOAP è un protocollo per lo scambio di messaggi tra componenti software, che permette di chiamare procedure remote (RPC Call, Remote Procedure Call). Richieste e risposte SOAP sono codificate con XML.},
	plural=SOAP
}

\newglossaryentry{DBMS}
{
	name=\glslink{DBMS}{DBMS},
	text=DBMS,
	sort=DBMS,
	description={Data Base Management System, sistema software progettato per la gestione (creazione, manipolazione, interrogazione) di basi di dati (database).},
	plural=DBMS
}

\newglossaryentry{RDBMS}
{
	name=\glslink{RDBMS}{RDBMS},
	text=RDBMS,
	sort=RDBMS,
	description={Relational Data Base Management System, DBMS basato sul modello relazionale.},
	plural=RDBMS
}


\newglossaryentry{framework}
{
	name=\glslink{framework}{framework},
	text=framework,
	sort=framework,
	description={Architettura software che include degli strumenti (classi, metodi) con lo scopo di semplificare lo sviluppo, facilitando così il lavoro del programmatore.},
	plural=framework
}


\newglossaryentry{wisp}
{
	name=\glslink{wisp}{WISP},
	text=WISP,
	sort=WISP,
	description={Acronimo di Windows (Server) - IIS - SQL Server - PHP. Viene utilizzato da WebPD per indicare lo stack tecnologico utilizzato dal \bookingEngine.},
	plural=WISP
}


\newglossaryentry{jquery}
{
	name=\glslink{jquery}{jQuery},
	text=jQuery,
	sort=jQuery,
	description={Una tra le più diffuse librerie Javascript, che agevola la manipolazione del \gls{dom}.},
	plural=jQuery
}

\newglossaryentry{dom}
{
	name=\glslink{dom}{DOM},
	text=DOM,
	sort=DOM,
	description={Acronimo di Document Object Model, rappresentazione in forma di albero di oggetti del contenuto della pagina HTML (Document) a cui si riferisce.}
}

\newglossaryentry{json}
{
	name=\glslink{json}{JSON},
	text=JSON,
	sort=JSON,
	description={Acronimo di JavaScript Object Notation, formato utilizzato per la rappresentazione di oggetti sotto forma di stringa.},
	plural=JSON
}


\newglossaryentry{ajax}
{
	name=\glslink{ajax}{AJAX},
	text=AJAX,
	sort=AJAX,
	description={Acronimo di Asyncronous Javascript And XML, tecnica che prevede lo scambio di dati in background tra server e browser, utilizzando richieste HTTP asincrone},
	plural=AJAX
}
 % database di termini
\makeglossaries


%**************************************************************
% Impostazioni di graphicx
%**************************************************************
\graphicspath{{immagini/}} % cartella dove sono riposte le immagini


%**************************************************************
% Impostazioni di hyperref
%**************************************************************
\hypersetup{
    %hyperfootnotes=false,
    %pdfpagelabels,
    %draft,	% = elimina tutti i link (utile per stampe in bianco e nero)
    colorlinks=true,
    linktocpage=true,
    pdfstartpage=1,
    pdfstartview=FitV,
    % decommenta la riga seguente per avere link in nero (per esempio per la stampa in bianco e nero)
    %colorlinks=false, linktocpage=false, pdfborder={0 0 0}, pdfstartpage=1, pdfstartview=FitV,
    breaklinks=true,
    pdfpagemode=UseNone,
    pageanchor=true,
    pdfpagemode=UseOutlines,
    plainpages=false,
    bookmarksnumbered,
    bookmarksopen=true,
    bookmarksopenlevel=1,
    hypertexnames=true,
    pdfhighlight=/O,
    %nesting=true,
    %frenchlinks,
    urlcolor=webbrown,
    linkcolor=RoyalBlue,
    citecolor=webgreen,
    %pagecolor=RoyalBlue,
    %urlcolor=Black, linkcolor=Black, citecolor=Black, %pagecolor=Black,
    pdftitle={\myTitle},
    pdfauthor={\textcopyright\ \myName, \myUni, \myFaculty},
    pdfsubject={},
    pdfkeywords={},
    pdfcreator={pdfLaTeX},
    pdfproducer={LaTeX}
}

%**************************************************************
% Impostazioni di itemize
%**************************************************************
\renewcommand{\labelitemi}{$\ast$}

%\renewcommand{\labelitemi}{$\bullet$}
%\renewcommand{\labelitemii}{$\cdot$}
%\renewcommand{\labelitemiii}{$\diamond$}
%\renewcommand{\labelitemiv}{$\ast$}


%**************************************************************
% Impostazioni di listings
%**************************************************************
\lstset{
    language=[LaTeX]Tex,%C++,
    keywordstyle=\color{RoyalBlue}, %\bfseries,
    basicstyle=\small\ttfamily,
    %identifierstyle=\color{NavyBlue},
    commentstyle=\color{Green}\ttfamily,
    stringstyle=\rmfamily,
    numbers=none, %left,%
    numberstyle=\scriptsize, %\tiny
    stepnumber=5,
    numbersep=8pt,
    showstringspaces=false,
    breaklines=true,
    frameround=ftff,
    frame=single
} 


%**************************************************************
% Impostazioni di xcolor
%**************************************************************
\definecolor{webgreen}{rgb}{0,.5,0}
\definecolor{webbrown}{rgb}{.6,0,0}


%**************************************************************
% Altro
%**************************************************************

\newcommand{\omissis}{[\dots\negthinspace]} % produce [...]

% eccezioni all'algoritmo di sillabazione
\hyphenation
{
    ma-cro-istru-zio-ne
    gi-ral-din
}

\newcommand{\sectionname}{sezione}
\addto\captionsitalian{\renewcommand{\figurename}{Figura}
                       \renewcommand{\tablename}{Tabella}}

\newcommand{\glsfirstoccur}{\ap{{[g]}}}

\newcommand{\intro}[1]{\emph{\textsf{#1}}}

%**************************************************************
% Environment per ``rischi''
%**************************************************************
\newcounter{riskcounter}                % define a counter
\setcounter{riskcounter}{0}             % set the counter to some initial value

%%%% Parameters
% #1: Title
\newenvironment{risk}[1]{
    \refstepcounter{riskcounter}        % increment counter
    \par \noindent                      % start new paragraph
    \textbf{\arabic{riskcounter}. #1}   % display the title before the 
                                        % content of the environment is displayed 
}{
    \par\medskip
}

\newcommand{\riskname}{Rischio}

\newcommand{\riskdescription}[1]{\textbf{\\Descrizione:} #1.}

\newcommand{\risksolution}[1]{\textbf{\\Soluzione:} #1.}

%**************************************************************
% Environment per ``use case''
%**************************************************************
\newcounter{usecasecounter}             % define a counter
\setcounter{usecasecounter}{0}          % set the counter to some initial value

%%%% Parameters
% #1: ID
% #2: Nome
\newenvironment{usecase}[2]{
    \renewcommand{\theusecasecounter}{\usecasename #1}  % this is where the display of 
                                                        % the counter is overwritten/modified
    \refstepcounter{usecasecounter}             % increment counter
    \vspace{10pt}
    \par \noindent                              % start new paragraph
    {\large \textbf{\usecasename #1: #2}}       % display the title before the 
                                                % content of the environment is displayed 
    \medskip
}{
    \medskip
}

\newcommand{\usecasename}{UC}

\newcommand{\usecaseactors}[1]{\textbf{\\Attori Principali:} #1. \vspace{4pt}}
\newcommand{\usecasepre}[1]{\textbf{\\Precondizioni:} #1. \vspace{4pt}}
\newcommand{\usecasedesc}[1]{\textbf{\\Descrizione:} #1. \vspace{4pt}}
\newcommand{\usecasepost}[1]{\textbf{\\Postcondizioni:} #1. \vspace{4pt}}
\newcommand{\usecasealt}[1]{\textbf{\\Scenario Alternativo:} #1. \vspace{4pt}}

%**************************************************************
% Environment per ``namespace description''
%**************************************************************

\newenvironment{namespacedesc}{
    \vspace{10pt}
    \par \noindent                              % start new paragraph
    \begin{description} 
}{
    \end{description}
    \medskip
}

\newcommand{\classdesc}[2]{\item[\textbf{#1:}] #2}